\documentclass{report}

\usepackage[utf8]{inputenc}
\usepackage[russian]{babel}
\usepackage[unicode=true]{hyperref}
\usepackage{paratype}
\setcounter{secnumdepth}{-1}

\begin{document}

\section{Лабораторная работа № 1}

\subsection{Задача}

Аналитическая задача — выявить характерную для некоторого фокусного
корпуса лексику (на фоне остального/референсного корпуса).
Этапы работы:

\begin{enumerate}
\item Чистка и нормализация текста: принять решения, влияющие на
  результат токенизации текстов и подсчета слов (считать ли
  пунктуацию, числа, смайлики, стоп-слова, нужна ли лемматизация,
  использовать ли и как подобрать пороги отсечения по частотности и т.п.).
\item Для выделения характерной лексики нужно последовательно
  использовать все рассмотренные нами на лекции способы оценки:
  простое отношение нормализованной частотности слов (по simple maths
  Адама Килгариффа), отношение правдоподобия (Dunning log-likelihood),
  PMI, weighted log-odds (по методу, реализованному в пакете tidylo).
\item Сравнить полученные всеми перечисленными способами списки
  наиболее характерной лексики и описать специфику фокусного корпуса,
\item а также охарактеризовать свои наблюдения (if any) относительно того,
  какие аспекты данных высвечивает каждая из использованных мер.
\end{enumerate}

\subsection{Данные}

В качестве данных для анализа нужно взять любую коллекцию текстов
(объем коллекции должен быть не менее нескольких тысяч слов, лучше
начинать от десятков тысяч).

В качестве коллекции можно использовать любой набор текстов на русском
языке (или другом языке, носителем которого вы являетесь). Это могут
быть тексты с любых интернет-платформ, литература, песни, тексты,
связанные с тематикой ваших интересов или курсовой работы, наконец,
ваши собственные тексты (переписка в мессенджерах, посты и т.п.).

Обязательное условие — у каждого сдавшего работу должен быть свой
корпус текстов. Если две или более работ будут выполнены на одних и
тех же данных, они не будут зачтены.

В собранной вами коллекции текстовых данных нужно выделить некоторое
подмножество текстов, объединенных каким-либо признаком. Это может
быть любой описательный признак, характеризующий текст (метаданные):
дата, происхождение текста, автор, жанр, тематическая рубрика и т.п.
Выделенное таким образом подмножество текстов станет фокусным
корпусом, а все остальные тексты коллекции — референсным.

\subsection{Отчет} 

Загружается на сервер в домашний каталог вашего пользователя. Отчет
должен состоять из двух файлов: lab1.Rmd и lab1.html (html-версия,
полученная с помощью knitr). Файлы будут автоматически собраны
скриптом, когда наступит дедлайн (по аналогии со сбором результатов
практических работ).

Если с загрузкой отчета на сервер возникли какие-то сложности,
допускается сдать отчет по электронной почте на адрес
kmaslinsky@hse.ru (с объяснением проблемы).


\subsubsection{Содержание отчета}

В отчете по результатам работы необходимо представить: 

\begin{enumerate}
\item Характеристику данных (принципы составления коллекции, ее объем
  в словах).
\item Постановку задачи — на каком основании выделена группа текстов,
  характерную лексику которой Вы определяете. 
\item Описание и обоснование принятых решений относительно
  нормализации и подсчета слов. 
\item Списки топ-слов (наиболее характерных слов группы) по каждой из
  рассчитанных мер специфичности, важно указать значения этих мер для
  каждого слова.
\item Общий вывод (интерпретацию): какие особенности выбранной группы
  текстов высвечивает анализ характерной лексики.
\item Комментарии по специфике отдельных мер: каковы отличия между
  мерами по характеру той специфической лексики, которую они выделяют.
\end{enumerate}

\subsubsection{Формат отчета}

В отчете помимо перечисленных выше содержательных моментов должны быть
представлены:

\begin{enumerate}
\item Код (весьма желательно оставить только релевантные, минимально
  необходимые для расчетов фрагменты кода, не загружая отчет полной
  историей всех успешных и неуспешных попыток).
\item Вывод результатов (таблицы топ-слов — обязательно, всякие
  графики и визуализации — добавить по вкусу).
\end{enumerate}

Отчет может быть оформлен в одном из двух форматов:
\begin{itemize}
\item[а)] в формате .Rmd, в этом случае необходимо приложить
  .html-файл с кодом, результатами и визуализацией, полученный с
  помощью knitr
\item[б)] «в разборе»: .R-скрипт с кодом (и комментариями), документ с
  текстом отчета, при необходимости приложенные файлы с результатами и
  визуализациями (в этом варианте сдавайте отчет по электронной
  почте). Если отчет содержит несколько файлов, пожалуйста, запакуйте их в архив
(и назовите его своим именем-фамилией, буду признателен).
\end{itemize}

Исходные данные (саму текстовую коллекцию) прикладывать к отчету \textbf{не нужно}.

\section{Критерии оценки}

\begin{itemize}
\item Полнота выполнения задания (посчитаны все метрики). 
\item Корректность расчетов (нет ошибок в формулах или сбоев в данных,
  которые приводят к неверным или бессмысленным результатам).
\item Полнота отчета: описаны все принятые решения, приведен код и
  результаты, все результаты прокомментированы, и дано обобщенное
  заключение о специфической лексике.
\item (Повышающий коэффициент) Глубина мысли: содержательная
  постановка задачи, интересные интерпретации, тонкие наблюдения,
  остроумные комментарии и другие виньетки, украшающие нашу
  академическую жизнь.
\item Неформальным, но неизбежным фактором, влияющим на общую оценку
  работы, будет удобочитаемость отчета.
\end{itemize}

\subsection{Сроки} 

Дедлайн — \textbf{17.10.2020 23:59}. Своевременная проверка работ, сданных
после дедлайна, \textbf{не гарантируется}. 

\subsection{Техподдержка}

Методологические вопросы, экзистенциальные сомнения, а также
технические проблемы давайте обсуждать на канале курса в Телеграм в
комментариях к посту, где объявлена лабораторная работа.

\end{document}

